% Please do not change the document class
\documentclass{scrartcl}

% Please do not change these packages
\usepackage[hidelinks]{hyperref}
\usepackage[none]{hyphenat}
\usepackage{setspace}
\doublespace

% You may add additional packages here
\usepackage{amsmath}

% Please include a clear, concise, and descriptive title
\title{Your Title Here}

% Please do not change the subtitle
\subtitle{COMP150 - CPD Report}

% Please put your student number in the author field
\author{1607804}

\begin{document}

\maketitle

\section{Introduction}
It is unclear to me at this point which pathway i want my career to take, however i do know i want it to be in the games industry.  From my weekly journal that has spanned the past eleven weeks I have identified five key skills that i have struggled with. These skills are as follows: Engaging in group problem solving, Understanding and considering the ideas of my peers, Prioritising tasks, engaging the audience and assertiveness in a team situation.


\section{Engaging in group problem solving}
To be part of any game development team i will need to be able to collectively problem solve. Throughout this semester I have had many opportunities to partake in a combined effort to fix a certain bug or help implement a certain feature. Upon reflection I noticed my natural response is to go off to one side load up my own version of the problem and try and solve it individually. This may have hindered my learning as I miss out on knowledge others in my team posses that i do not. In addition trying to solve a problem that others in the team are better qualified for will result in a waste of time and resources. To overcome this obsticle 

Write about 200 words about. Remember, this is should be reflective and personal to you. Justify the relevance and importance of each of these skills with insight into your personal goals and personal circumstances. Assess your application of the skill throughout the semester and critically reflect on upon their impact it has had on your work and the challenges/obstacles. Acknowledge difficulties. Then, suggest how to overcome the challenge/obstacle in relation to a SMART action. When planning such actions, do not be too general. Consider SMART actions:
specific; measurable; achievable; relevant; and time-bound. Ensure that your proposed action for future development meets all five of these criteria.

\section{Second Key Skill}

Write about 200 words. As above.

\section{Third Key Skill}

Write about 200 words. As above.

\section{Fourth Key Skill}

Write about 200 words. As above.

\section{Fifth Key Skill}

Write about 200 words. As above.

\section{Conclusion}

Write your conclusion here. Though the conclusion should be brief, no more than 100 words, it should do more than merely summarise the report. Focus on the five SMART actions that you intend to take in order to overcome any challenges and/or obstacles. Contextualise how this will help you towards your intended career goal and how this may improve your project for the next semester.

\bibliographystyle{ieeetran}
\bibliography{references}

\end{document}
