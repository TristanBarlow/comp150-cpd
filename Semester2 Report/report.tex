% Please do not change the document class
\documentclass{scrartcl}

% Please do not change these packages
\usepackage[hidelinks]{hyperref}
\usepackage[none]{hyphenat}
\usepackage{setspace}
\doublespace

% You may add additional packages here
\usepackage{amsmath}

% Please include a clear, concise, and descriptive title
\title{CPD Report}

% Please do not change the subtitle
\subtitle{COMP160 - CPD Report}

% Please put your student number in the author field
\author{1607804}

\begin{document}

\maketitle

\section{Patience when dealing with uncooperative peers}This semester during the second sprint I found myself being impatient with peers after their unwillingness to participate in agile activities and their lack of contribution to the vision of the project became apparent. When I am working in the industry, not having patients with my peers may lead to bad relations in the workplace. If I show more patients it will allow me to better integrate into new teams. Not being patient with my peers this semester has made some strained situations during PO meetings and stand-ups. It made me less motivated to work on the group project, reducing the amount of work I have done. In the future, I will overcome this by getting to know the uncooperative person, asking why they're not participating and try to understand why they're acting the way they are. I will measure this by the cohesion of the teams I work in, in the future. This should something I should aim to do for the foreseeable future. 

\section{Maintaining good lines of communication to my team}
Maintaining good lines of communication is not only in keeping with an Agile workflow, but will also help with the skill mentioned above. Showing the ability to be a good communicator to my team is a very desirable skill as it aids in productivity and transparency of the team as a whole. This may also open more job opportunities as displaying good communication skills may also make me a more viable pick for promotional events. During an illness I suffered in the third sprint of this semester, I found that it had been two days since I last go in contact with another member of my team. This resulted in me being behind and unaware of some of the changes made to the game project. In one such case, it meant I had developed a mechanic to find it already implemented in the game. To avoid this mistake again, I will check the preferred messaging service at least once a day, regardless of my own well-being (within reason). During our next group project, I plan to measure the number of days that I go without contact. 

\section{Managing task resources (Trello)}
By task resources, I mean the form of management system used for task distribution. Correct Management of whichever task system used by the company I will be employed in is crucial to maintain good workflow and to keep any stakeholders who have access to the system updated on what I am currently working on. In addition, it will allow the communication between the team improve particularly on the topic of what someone is currently working. In this group project, we used Trello. I found myself choosing to remember tasks that I have to do by writing them on my phone or just relying on my memory. I would later add them to Trello once I had finished. I found that Trello became more of a way to say what I have done, not what I am doing/going to do. This resulted in confusion within the team to what I was actually working on. This became even more apparent during stand up meetings. This malpractice has a negative impact not only on me but on my team. To target this problem, I will use whichever system as my first go to when creating a to-do list. In addition, if it is a digital task board, I will aim to check in at least every day including weekends. With this kind of skill, I believe it is about getting to know the tools in use and in time it will become routine. By the end of next semester, I hope to be proficient with Trello and think of apps like Trello as my preferred way of managing my to-do list.

\section{Efficiently contribute to the concept}
When programming a mechanic for a game, a lot of the decisions about how it will play is down to the way it is programmed. Thus, the way the mechanic will feel in game is partly decided by the programmer.  Not understanding the concept may result in a mechanic that will fit the specification but not 'fit' with the game. Creating mechanics or gameplay features that feel wrong may result in wasted man hours. Wasted man hours means wasted money and stakeholders of any business will not like the sound of wasted money. This semester, for the first sprint, the concept I had in mind for our game was different to others on my team. This negatively impact my work as I made content that did not feel 'right'. When I was asked to produce a mechanic that allowed for PvP, my solution did not fit with the assets produced by the rest of the team. To avoid wasted working hours due to not understanding the vision of the game I will begin by reading any design documents provided. Where design documents are not present I will ask and have my teammates explain the concept until I fully understand. This will reduce the risk of the content I create not fitting with the vision of the game. In addition focusing more on paired programming will dilute any misunderstanding of the concept as there will be more than one perspective in the creation of content.

\section{Teaching others}
There may be a time in the industry where a college, be it an artist seeking technical support or a fellow programmer. The ability to help the college with his/her problem while desirable to have is usually only a short-term fix. If I can go above and beyond by teaching them the whys and the hows will benefit all those who ask help of me. It means if colleges run into the same problem again, they will be able to solve it themselves saving myself time and therefore the company money. This semester one of the game designers on my team asked me how the classes responsible for the enemy spawners work. While I was not the creator of these I did understand how they worked. Due to my inability to teach them how it worked, it led to massive code duplication as the game designer sort their own solution. If I were able to teach better this would not have happened and my college would not have wasted so much of their time. To get better at teaching I will ask feedback after I have attempted to teach. By keeping a record of the feedback, be it from surf-coaching or programming, I will then use this feedback going forward to the next time. I hope by this time next year my ability to teach others will have improved greatly.

\bibliographystyle{ieeetran}
\bibliography{references}

\end{document}
