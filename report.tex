% Please do not change the document class
\documentclass{scrartcl}

% Please do not change these packages
\usepackage[hidelinks]{hyperref}
\usepackage[none]{hyphenat}
\usepackage{setspace}
\doublespace

% You may add additional packages here
\usepackage{amsmath}

% Please include a clear, concise, and descriptive title
\title{Your Title Here}

% Please do not change the subtitle
\subtitle{COMP150 - CPD Report}

% Please put your student number in the author field
\author{1607804}

\begin{document}

\maketitle

\section{Introduction}
It is unclear to me at this point which pathway i want my career to take, however i do know i want it to be in the games industry.  From my weekly journal that has spanned the past eleven weeks I have identified five key skills that i have struggled with. These skills are as follows: Engaging in group problem solving, Understanding and considering the ideas of my peers, Prioritising tasks, engaging the audience and assertiveness in a team situation.


\section{Engaging in group problem solving}
To be part of any game development team i will need to be able to collectively problem solve. Throughout this semester I have had many opportunities to partake in a combined effort to fix a certain bug or help implement a certain feature. Upon reflection I noticed my natural response is to go off to one side load up my own version of the problem and try and solve it individually. This may have hindered my learning as I miss out on knowledge others in my team posses that i do not. In addition trying to solve a problem that others in the team are better qualified for will result in a waste of time and resources. In future I will lead group problem solving activites. By taking notes during these activites, I can record the quantity and contents of the group problem solving for later reference. Engaging in this will help me develop a larger range of skills that will allow me to work more effective in the games industry. I hope to achieve this action by the end of the next semester.

Write about 200 words about. Remember, this is should be reflective and personal to you. Justify the relevance and importance of each of these skills with insight into your personal goals and personal circumstances. Assess your application of the skill throughout the semester and critically reflect on upon their impact it has had on your work and the challenges/obstacles. Acknowledge difficulties. Then, suggest how to overcome the challenge/obstacle in relation to a SMART action. When planning such actions, do not be too general. Consider SMART actions:
specific; measurable; achievable; relevant; and time-bound. Ensure that your proposed action for future development meets all five of these criteria.

\section{Understanding and considering the ideas of my peers.}
In the games industry I will have to work on games who's  concept does not belong to me. The ability to understand and show consideration for the thoughts and ideas of my peers, is important to make sure there is no conflict between the ideas of the team and myself. During pair programming exercises this semester I found myself disregarding the ideas of my partner in favour for my own. The reason for my disregard being my inability to immediately understand my peer. In a professional environment not being able to understand a college, could result in bad working practice (hard to maintain code etc.). Capture the ideas of other group members by writing them down and create sufficient space to consider them. By ticking of the ideas after consideration I can measure my progress through all the ideas suggest by my peers. Showing the ability to understand colleges ideas will increase my employability. Also considering both the creative ideas and the technical details will help me develop both my technical and creative skills. I aim to complete this action by the end of my time at university. 

\section{Prioritising tasks}
A large part of the creation of the game is picking tasks you will be able to complete within the time given and choosing what tasks to do first. Choosing the wrong task to complete first could create a backlog for other members of the team an inhibit they're ability to complete their tasks. My journal highlights a few times where this has been a problem, one of which is about an assignment deadline. I found myself completing assignments with deadlines that were weeks away to the exclusion of assignments with deadlines days away. Task management like this caused inconsistencies with my work, some being of a high standard and others being low. I will overcome this by creating a`to do list' in order of importance so that my mind stays focused. One way I can measure my success is by the quality the work i produce in this way. Another is through my peers feedback, I will ask them if the order of which i completed my tasks has anyway stop them completing theirs. I believe mastering prioritisation will not only make my work better but will allow my team mates room to produce better work. I will complete this action by the end of the first sprint of our group project in semester two.


\section{Engaging the Audience l}



\section{Fifth Key Skill}

Write about 200 words. As above.

\section{Conclusion}

Write your conclusion here. Though the conclusion should be brief, no more than 100 words, it should do more than merely summarise the report. Focus on the five SMART actions that you intend to take in order to overcome any challenges and/or obstacles. Contextualise how this will help you towards your intended career goal and how this may improve your project for the next semester.

\bibliographystyle{ieeetran}
\bibliography{references}

\end{document}
