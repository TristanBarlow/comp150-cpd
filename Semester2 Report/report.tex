% Please do not change the document class
\documentclass{scrartcl}

% Please do not change these packages
\usepackage[hidelinks]{hyperref}
\usepackage[none]{hyphenat}
\usepackage{setspace}
\doublespace

% You may add additional packages here
\usepackage{amsmath}

% Please include a clear, concise, and descriptive title
\title{CPD Report}

% Please do not change the subtitle
\subtitle{COMP160 - CPD Report}

% Please put your student number in the author field
\author{1607804}

\begin{document}

\maketitle

\section{Introduction}


\section{Patience when dealing with un co-operative peers}
- I have been impatient with peers after their unwillingness to co-operate. In industry this would be bad because it could lead to bad relations in the work place. Being more patient will make me more cohesive in teams and more employable.

- Not being patient with my peers this semester has made some strained situations during PO meetings and standups. Its made me less motivated to work on the group project, reducing the amount of work i have done.

-I will over come this by getting to know the un cooperative person, asking why they're not participating and try to understand why they're acting the way they are. I will measure this by the cohesion of the teams i work in, in the future. This should something I should aim to do for the forseeable furture. 

\section{Maintaining good lines of comunication to my team}
-Mainting good lines of commuination is not only in keeping with an Agile workflow but also helps somewhat with the skill mentioned above. Showing the ability to be a good comunicator to my team is a very desirable skill as it aids in productivity and transparency of the team as a whole

-At times this semester where I have been ill I have had no contact with my team for at most 2 days at a time. This resulted in me being behind and unaware of some of the changes made to the game project. In one such case it meant I had developed a mechanic to find it already implemented in game.

-I will over come this by checking the prefered messaging service at least once a day regardless of my own wellbeing. 


\section{Managing task resources (Trello)}
-By task resources I mean whatever form of managment system used for task distrubtion. Correct Managment of whicheever task system used by the company I will be employed in is crucial to maintain good workflow and to keep any stakeholders who have accsess to the system updated on what i am currently working on.

- In this group project we used Trello. I found myself choosing to remember tasks that I have to do by writing them in my phone or just relying on my memory. I would later add them to Trello once I had finished. This resulted in people confused to what i was actually working on, especially during times where we didnt have stand up meetings. This more negativly impacted my team then me personally.

- To target this, I will use whichever system as my first go to when creating a "to do" list. In addition If it is a digital task board I will like above check in at least everyday including weekends. This should get me into a routine of using the system. 


\section{Efficently contribute to the concept}
-When programming a mechanic for a game, a lot of the decisions about how it will play is down to the way it is programmed. Thus the way the mechanic will feel is can be decieded by the programmer.  Not understanding the concept may result in a mechanic that will fit the specification but not 'fit' with the game.

-This semester for the first sprint the concept i had in mind for our game was different to other in the group. This negatively impact my work as I made content that didnt feel 'right'. 

- By reading any design doccuments provided, I will better understand the game I am working on. This will reduce the risk of the content I create not fitting with the vision of the game. 


\section{Teaching others.}
There may be a time in industry where college be it an artist seeking techincal support or a fellow programmer seeking support.


\section{Conclusion}


\bibliographystyle{ieeetran}
\bibliography{references}

\end{document}
